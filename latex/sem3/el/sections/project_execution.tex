\section{Project Execution}

\subsection{Planning and Design}
The initial planning phase focused on establishing the core architecture and design patterns:

\begin{enumerate}
    \item \textbf{System Architecture Design}
    \begin{itemize}
        \item Microservices-based architecture for scalability
        \item Event-driven design for real-time updates
        \item Layered security approach for data protection
        \item Modular component structure for maintainability
    \end{itemize}

    \item \textbf{Database Schema Design}
    \begin{itemize}
        \item PostgreSQL with pgvector extension for embeddings
        \item Structured tables for user data and health records
        \item Time-series data storage for sensor readings
        \item Efficient indexing for quick retrieval
    \end{itemize}

    \item \textbf{API Design}
    \begin{itemize}
        \item RESTful endpoints for CRUD operations
        \item WebSocket connections for real-time data
        \item Authentication middleware integration
        \item Rate limiting and request validation
    \end{itemize}
\end{enumerate}

\subsection{Implementation}
The implementation phase was executed in several stages:

\begin{enumerate}
    \item \textbf{Frontend Development}
    \begin{itemize}
        \item Implementation of responsive dashboard layouts
        \item Creation of reusable UI components
        \item Integration of data visualization tools
        \item Development of form validation logic
    \end{itemize}

    \item \textbf{Backend Services}
    \begin{itemize}
        \item Setup of FastAPI server infrastructure
        \item Implementation of database models and migrations
        \item Development of authentication services
        \item Creation of data processing pipelines
    \end{itemize}

    \item \textbf{AI Integration}
    \begin{itemize}
        \item Implementation of RAG pipeline using LangChain
        \item Integration of OpenAI's GPT models
        \item Development of context management system
        \item Creation of health insights generation logic
    \end{itemize}

    \item \textbf{Sensor Integration}
    \begin{itemize}
        \item Arduino sensor programming and calibration
        \item Implementation of data collection protocols
        \item Development of real-time processing logic
        \item Creation of alert monitoring system
    \end{itemize}
\end{enumerate}

\begin{figure}[H]
    \centering
    \includegraphics[width=0.8\textwidth]{figures/implementation_components.png}
    \caption{Implementation Components and Integration}
\end{figure}

\subsection{Key Implementation Challenges}

\begin{enumerate}
    \item \textbf{Data Processing Optimization}
    \begin{itemize}
        \item Handling large volumes of health records
        \item Optimizing embedding generation process
        \item Managing real-time data streams
        \item Implementing efficient caching strategies
    \end{itemize}

    \item \textbf{Security Implementation}
    \begin{itemize}
        \item Ensuring HIPAA compliance
        \item Implementing end-to-end encryption
        \item Managing secure data transmission
        \item Handling user authentication securely
    \end{itemize}

    \item \textbf{Integration Challenges}
    \begin{itemize}
        \item Coordinating multiple service interactions
        \item Managing API rate limits
        \item Handling network latency issues
        \item Ensuring data consistency across services
    \end{itemize}
\end{enumerate} 