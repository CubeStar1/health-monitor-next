\section{Testing Strategy}
\subsection{Unit Testing}
Based on the project structure:

\begin{enumerate}
    \item \textbf{Frontend Component Testing}
    \begin{lstlisting}[language=typescript]
    // RecordDetails.test.tsx
    describe('RecordDetails', () => {
        it('should render health record details', () => {
            const record = {
                id: '123',
                file_name: 'test.pdf',
                analysis: {
                    text_content: 'Test content',
                    indicators: []
                }
            };
            
            render(<RecordDetails record={record} />);
            expect(screen.getByText('test.pdf')).toBeInTheDocument();
        });
    });
    \end{lstlisting}

    \item \textbf{Hook Testing}
    \begin{lstlisting}[language=typescript]
    // useSensorData.test.ts
    describe('useSensorData', () => {
        it('should subscribe to sensor updates', async () => {
            const { result } = renderHook(() => useSensorData());
            expect(result.current.isLoading).toBe(true);
            // Test real-time updates
        });
    });
    \end{lstlisting}
\end{enumerate}

\subsection{Integration Testing}
\begin{itemize}
    \item \textbf{API Integration Tests}
    \begin{itemize}
        \item Document upload and processing flow
        \item RAG pipeline functionality
        \item Sensor data integration
        \item Real-time updates
    \end{itemize}

    \item \textbf{End-to-End Testing}
    \begin{itemize}
        \item User authentication flow
        \item Health record management
        \item Dashboard functionality
        \item Chat interface
    \end{itemize}
\end{itemize} 