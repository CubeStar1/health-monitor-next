\section{System Architecture and Design}

\subsection{System Architecture Overview}

The system architecture comprises a modular IoT-based health monitoring framework integrated with a cloud backend and an AI-powered frontend interface. The architecture includes microcontrollers, biomedical sensors, communication modules, a database, backend APIs, and a real-time frontend dashboard. Sensor data is transmitted via Wi-Fi to a cloud database where it is processed, visualized, and queried using natural language.

\begin{figure}[h!]
\centering
\includegraphics[width=\textwidth]{images/feature_implementation.png}
\caption{System Architecture Diagram}
\end{figure}

\subsection{Hardware Components and Configuration}

\textbf{Components and Sensors Used:}

\begin{itemize}
    \item \textbf{Microcontroller:}
    \begin{itemize}
        \item \textbf{Arduino UNO R3} – Acts as the core controller for sensor interfacing and data collection.
    \end{itemize}

    \item \textbf{Wi-Fi Module:}
    \begin{itemize}
        \item \textbf{ESP8266} – Ensures wireless connectivity and enables real-time data transfer to the backend.
    \end{itemize}

    \item \textbf{Biomedical Sensors:}
    \begin{itemize}
        \item \textbf{MAX30102} – Captures \textbf{SpO2} and \textbf{heart rate} data using IR and red LED photodetectors.
        \item \textbf{AD8232} – Acquires real-time \textbf{ECG signals}, enabling heart activity tracking through electrode pads.
        \item \textbf{SEN-11574} – Detects \textbf{heartbeats} through finger contact with analog output.
        \item \textbf{LM35} – Measures \textbf{body temperature} with high accuracy via linear voltage output.
    \end{itemize}

    \item \textbf{Display Module:}
    \begin{itemize}
        \item \textbf{OLED Display (e.g., SSD1306)} – Visual feedback module for showing sensor values in real-time.
    \end{itemize}
\end{itemize}

\begin{figure}[h!]
\centering
\includegraphics[width=\textwidth]{images/sensor_integration.png}
\caption{IoT Sensor Setup and Configuration}
\end{figure}

\subsection{Technology Stack}

\textbf{Tech Stack Overview:}

\begin{itemize}
    \item \textbf{Hardware:} Arduino UNO R3, ESP8266, MAX30102, AD8232, LM35, SEN-11574, OLED Display
    \item \textbf{Firmware:} Arduino IDE (C++), sensor-specific libraries
    \item \textbf{Backend:} FastAPI, PostgreSQL (Supabase), MQTT/HTTP protocol
    \item \textbf{Frontend:} Next.js, TailwindCSS, shadcn/ui, Chart.js or Recharts
    \item \textbf{AI/NLP:} OpenAI GPT-4, LangChain, LangGraph, Cohere (for embeddings)
    \item \textbf{Storage \& Auth:} Supabase (Auth, RLS, pgvector for embeddings, file storage)
\end{itemize}

\subsection{Circuit Diagrams and Pin Configurations}
% Please add relevant circuit diagrams and pin configuration details here.
% Example: \begin{figure}[h!]
% \centering
% \includegraphics[width=0.8\textwidth]{images/circuit_diagram.png}
% \caption{Overall Circuit Diagram}
% \end{figure}

% Pin connections for Arduino UNO R3:
% - MAX30102: SDA -> A4, SCL -> A5
% - AD8232: OUTPUT -> A0, LO+ -> D10, LO- -> D11
% - ... and so on for other sensors and modules.
