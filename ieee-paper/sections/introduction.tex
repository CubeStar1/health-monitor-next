\section{Introduction}

Healthcare data management faces increasing complexity with the proliferation of digital health records and real-time sensor data [3]. Recent advances in Retrieval-Augmented Generation (RAG) have shown promising results in processing medical records [1, 7]. Traditional approaches struggle to effectively integrate and analyze these diverse data sources [5], particularly when combining unstructured medical records with structured sensor data. This paper presents HealthHub, a novel platform that addresses these challenges through two key AI components: a Retrieval-Augmented Generation (RAG) pipeline for medical records and an SQL agent for sensor data analysis.

\subsection{Problem Statement}
Current healthcare systems face several critical challenges:
\begin{itemize}
\item Difficulty in processing and analyzing unstructured medical documents
\item Limited natural language interfaces for querying health sensor data
\item Lack of integration between historical records and real-time monitoring
\item Complexity in maintaining context across different data types
\end{itemize}

\subsection{Objectives}
The primary objectives of this research include:
\begin{itemize}
\item Development of a RAG pipeline for intelligent medical record processing
\item Implementation of an SQL agent for natural language sensor data queries
\item Integration of both systems for comprehensive health data analysis
\item Achievement of high performance and accuracy in data processing
\end{itemize}

\subsection{Contributions}
This paper makes the following contributions:
\begin{itemize}
\item A novel RAG pipeline implementation for medical record analysis
\item An innovative SQL agent design for sensor data interpretation
\item An integrated approach to handling structured and unstructured health data
\item Performance benchmarks for healthcare data processing systems
\end{itemize}

The integration of SQL agents with healthcare systems has demonstrated significant potential \cite{troy2023}, especially when combined with RAG technologies \cite{omrani2024}. Our work builds upon these foundations while addressing the challenges identified in recent systematic reviews \cite{amugongo2024, bora2024}. 