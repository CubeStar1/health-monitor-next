\section{Conclusion}

\subsection{Key Achievements}
This paper presented HealthHub, a novel health data management platform that successfully integrates two key AI components, building on recent advances in healthcare AI [3, 5]:

\begin{itemize}
\item A RAG pipeline that enables context-aware querying of medical records [1, 9]
\item An SQL agent that provides natural language interface to sensor data [14, 15]
\end{itemize}

The system demonstrated robust performance metrics:
\begin{itemize}
\item Sub-100ms response times for both RAG queries and SQL operations
\item 99.9\% system uptime during extended testing
\item Successful processing of 10,000+ medical documents and 1M+ sensor readings
\item High user satisfaction rates exceeding 90\% for both components
\end{itemize}

\subsection{Future Work}
Several promising directions for future development have been identified in recent research [11, 4]:

\begin{itemize}
\item \textbf{Enhanced Integration:} Deeper integration between RAG and SQL components for more complex hybrid queries
\item \textbf{Performance Optimization:} Further reduction in response times through advanced caching strategies
\item \textbf{Extended Capabilities:} Integration with additional medical sensors and data sources
\item \textbf{Security Enhancements:} Implementation of additional HIPAA compliance measures
\end{itemize}

\subsection{Impact}
The success of HealthHub demonstrates the potential of combining RAG pipelines with SQL agents in healthcare applications [10]. This approach provides a foundation for future developments in healthcare technology, addressing key challenges identified in recent systematic reviews [16].

The platform's ability to process both structured and unstructured medical data while maintaining high performance and accuracy makes it a valuable contribution to the field of healthcare technology. 